\documentclass{article}
\usepackage[utf8]{inputenc}
\usepackage{amsmath}
\usepackage{graphicx}
\usepackage{float}
\begin{document}
\begin{titlepage}
\centering
{\bfseries\LARGE Fundación Universitaria Konrad Lorenz\par}
\vfill
\noindent\hrulefill \\
{\scshape\Huge Project Mathematical Modeling\par}
\vspace{0.5cm}
{\scshape\Large Modelación y Simulación I \par}
\vspace{0.5cm}
{\scshape\Large 2025-I \par} 
\noindent\hrulefill \\
\vfill
{\bfseries\Large Profesor: Julian Orlando Jimenez Cardenas\par}
\vfill
{\Large Briam Alexander Palma Murillo - 614232005\par}
{\Large Gabriela Castañeda Monroy- 614232709\par}
\vfill
{\large Marzo de 2025 \par} 
\end{titlepage}
\setcounter{page}{1}
% Aqui iria la parte de bases de datos
\section {Requisitos de Datos}
\subsection{Simple thermodynamic model of thermostats for a liquid
into a tank: an analytical approach 
 }
 \begin{itemize}
    \item https://0310a08hk-y-https-link-springer-com.konrad.metaproxy.org/article/10.1007/s10665-024-10416-5
    
    \item Imagen de referencia:
\end{itemize}

\begin{figure}[h]
    \centering
    \includegraphics[width=1\textwidth]{CAP.jpeg}
    \caption{Descripción de la imagen}
    \label{fig:cap}
\end{figure}
 \subsection{Descripción de los Datos}
\begin{itemize} 

    \item \( T(t) \): Temperatura del fluido en el tanque.
    \item \( T_{\text{env}} \): Temperatura del ambiente.
    \item \( T'(t) \): Temperatura de la resistencia calefactora.
\end{itemize}


\begin{itemize}
    \item \( \kappa_1 \): Coeficiente de transferencia de calor entre el fluido y el ambiente.
    \item \( \kappa_2 \): Coeficiente de transferencia de calor entre la resistencia y el fluido.
    \item \( S_1 \): Área de contacto térmico entre el fluido y el ambiente.
    \item \( S_2 \): Área de contacto térmico entre la resistencia y el fluido.
\end{itemize}

\begin{itemize}
    \item \textbf{Tiempo característico de enfriamiento del fluido hacia el ambiente:}
    \begin{equation}
        \tau_1 = \frac{c_f m_f}{\kappa_1 S_1}
    \end{equation}
    \item \textbf{Tiempo característico de transferencia de calor de la resistencia al fluido:}
    \begin{equation}
        \tau_2 = \frac{m_f c_f}{\kappa_2 S_2}
    \end{equation}
\end{itemize}

Estos valores se utilizan en las ecuaciones diferenciales del modelo para describir la disipación de calor en el sistema según la Ley de Enfriamiento de Newton.
\section{Enfoque Matemático}
\subsection{Ley de Enfriamiento de Newton}
La tasa de cambio de la temperatura del café se modela mediante la ecuación diferencial:
\[
\frac{dT}{dt} = -k(T - T_{\text{ambiente}})
\]
donde:
\begin{itemize}
    \item \(T(t)\): Temperatura del café en el instante \(t\) (en °C).
    \item \(T_{\text{ambiente}}\): Temperatura ambiente constante.
    \item \(k\): Constante de enfriamiento (en \(s^{-1}\)).
\end{itemize}
La solución analítica de la ecuación es:
\[
T(t) = T_{\text{ambiente}} + (T_0 - T_{\text{ambiente}}) e^{-kt}
\]

\subsection{Justificación del Método}
\begin{itemize}
    \item La Ley de Newton es ideal para establecer con que rapidez se enfria un objeto respecto a la temperatura del ambiente y la temperatura con la que inica el objeto.
\end{itemize}

\subsection{Suposiciones del Metodo}
\begin{itemize}
    \item La temperatura ambiente permanece constante.
    \item No hay pérdida de calor por evaporación o convección forzada.
    \item La constante \(k\) dependera del material y la forma de la taza pero se va a ajustar a un valor de 0.1
\end{itemize}

\section{Plan de Implementación}
\subsection{Herramientas}
\begin{itemize}
    \item Uso de las librerías \texttt{numpy} y \texttt{matplotlib.pyplot}.
    \item Por medio de Jupyter Notebook.
\end{itemize}

\subsection{Pasos Clave}
\begin{enumerate}
    \item \textbf{Definir parámetros}: \(T_0\), \(T_{\text{ambiente}}\), \(k\).
    \item \textbf{Generar datos teóricos}: Usar la solución analítica para calcular \(T(t)\).
    \item \textbf{Visualización}: Graficar \(T(t)\) vs \(t\) para un intervalo de 60 minutos.
    \item \textbf{Validación}: Comparar la curva teórica con datos experimentales (si se dispone de ellos).
\end{enumerate}
\section{Resultados Esperados}
\begin{itemize}
    \item Determinar el tiempo para que el café alcance \(60^\circ C\) (apto para consumo).
    \item Evaluar cómo cambios en \(k\) afectan la velocidad de enfriamiento.
    \item Se espera obtener una curva de enfriamiento que muestre cómo la temperatura del café disminuye con el tiempo.
\end{itemize}
\end{document}